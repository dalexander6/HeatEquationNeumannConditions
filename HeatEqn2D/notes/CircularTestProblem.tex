\documentclass[a4paper,10pt]{article}
%\documentclass[a4paper,10pt]{scrartcl}
\usepackage{amssymb}
\usepackage[utf8]{inputenc}
\usepackage{verbatim}
\usepackage{amsthm}
\usepackage{amsmath}
\usepackage{graphicx}
\title{A test problem}
\author{Damon Alexander}

\pdfinfo{%
  /Title    ()
  /Author   ()
  /Creator  ()
  /Producer ()
  /Subject  ()
  /Keywords ()
}
\newcommand{\norm}[1]{\left|\left|#1\right|\right|}
\newcommand{\normInf}[1]{\left|\left|#1\right|\right|_\infty}
\newcommand{\normBoundary}[1]{\left|\left|#1\right|\right|_{L^2(\Gamma_2)}}

\newcommand{\normI}[1]{\left|\left|#1\right|\right|_{L^\infty(I)}}
\newcommand{\normH}[1]{\left|\left|#1\right|\right|_{H^2}}
\newcommand{\normh}[1]{\left|\left|#1\right|\right|_{H^1}}
\newcommand{\dx}[2]{\frac{\partial #1}{\partial x_#2}}
\newcommand{\normL}[2]{\left|\left|#1\right|\right|_{L^{#2}}}

\newcommand{\dn}[1]{\frac{\partial #1}{\partial \eta}}
\newcommand{\deriv}[1]{\frac{d #1}{dx}}
\newcommand{\derivv}[1]{\frac{d^2 #1}{dx^2}}
\newcommand{\der}[2]{\frac{\partial #1}{\partial #2}}
\newcommand{\p}{\mathbb{P}}
\newcommand{\e}{\mathbb{E}}
\newcommand{\var}{\mathbb{V}ar}
\newcommand{\R}{\mathbb{R}}
\newcommand{\Z}{\mathbb{Z}}
\newcommand{\Q}{\mathbb{Q}}
\newcommand{\B}{\mathcal{B}(\mathbb{R})}
\newcommand{\N}{\mathbb{N}}
\newcommand{\supp}[0]{\mbox{supp}}

\begin{document}
\maketitle

We generate a toy problem to test our solver against.  We try to solve in polar
\begin{align*}
	u_t &= \Delta u \mbox{ for } r < 1\\
	u_r(1,\theta,t) &= 0 \\ 
	u(r, \theta, 0) &= f(r, \theta) \mbox{ for } r < 1.
\end{align*}
For now we will assume that $f$ actually has radial symmetry.  Our first step is to make the ansatz $u(r, \theta,t) = R(r) T(t) \Theta(\theta)$.  This transforms our equation into 
\begin{align*}
	T' R \Theta =  R'' T \Theta + \frac{1}{r} R' \Theta T + \frac{1}{r^2} \Theta'' R T.
\end{align*}
Now we divide through by $R \Theta T$ to obtain
\begin{align*}
	\frac{T'}{T} = \frac{R''}{R} + \frac{R'}{r R} + \frac{ \Theta''}{r^2 \Theta}.
\end{align*}
Then since the LHS only depends on $t$ and the RHS only depends on $r$ and $\theta$, they must both be constant, which we call $- \lambda^2$ (there are reasons to expect it should be negative).  Thus
\begin{align*}
	T(t) = C e^{- \lambda^2 t}.
\end{align*}
For an as-yet unclear reason we make the ansatz
\begin{align*}
	\Theta'' / \Theta = -n^2.
\end{align*}
which has solutions $a_n \cos(n \theta) + b_n \sin(n\theta)$.  Then we find that after multiplying by $r^2 R$,
\begin{align*}
  r^2 R'' + r R' + \left( \lambda^2 r^2 - n^2\right) R = 0	.
\end{align*}
After changing variables $r \to x /\lambda$, we find that
\begin{align*}
  x^2 R'' + x R' + \left( x^2 - n^2\right) R = 0	.
\end{align*}
Thus $R(x)$ is a Bessel function of the first kind (since we want it to be finite at the origin), that is,
\begin{align*}
	R(r) = J_n(r \lambda).
\end{align*}
We now have two degrees of freedom: picking $n$ and $\lambda$.  In order to satisfy $R'(1) = 0$ we require that $\lambda = \lambda^n_m$ where $\{\lambda^n_m\}$ are the roots of $J_n'$.  Thus our final solution is
\begin{align*}
	u(r, \theta, t) = \sum_{m, n \ge 0} c_{m,n} [a_n \cos (n \theta) + b_n \sin(n \theta)] J_n( \lambda^n_m r) e^{-(\lambda^n_m )^2 t}.
\end{align*}
TODO: Figure out how this actually works.  It seems like we have more constants than should be necessary but maybe not?

\subsection*{A special case}
Okay, so let's just suppose
\begin{align*}
	f(r, \theta) = \cos(\theta) J_0(\lambda^0_0 r).
\end{align*}
Then our solution will be
\begin{align*}
	u(r, \theta, t) = \cos( \theta) J_0(\lambda^0_0 r) e^{-(\lambda^0_0)^2 t}.
\end{align*}

\subsection{A better special case} % (fold)
\label{sub:A better special case}
It turns out that $\cos(\theta)$ is discontinuous at the origin.  Since $J_0$ is non-zero at the origin, this is annoying.  From looking at pictures of Bessel functions, it looks like $J_3$ would be a better fit: according to wikipedia, for small arguments
 \begin{align*}
 	J_\alpha \approx z^\alpha.
 \end{align*}
Thus taking $J_3$ should make things more differentiable.  This gives
\begin{align*}
	u(r, \theta, t) = \cos(3 \theta) J_3(\lambda^3_0 r) e^{-(\lambda^3_0)^2 t}.
\end{align*}

% subsection A better special case (end)







\end{document}
